\documentclass[a4paper,10pt,titlepage,oneside]{report}

\usepackage[magyar]{babel}
\usepackage[utf8]{inputenc}

\title{\textbf{Webalapú jegyzőkönyvkészítő és ülésszervezést segítő alkalmazás}}
\author{Fási Gábor}
\date{}

\frenchspacing

\renewcommand{\thesection}{\arabic{section}}

\begin{document}
\setcounter{chapter}{1}

\maketitle

\section{A feladat felvázolása}

Témám egy olyan rendszer elkészítése, mely böngészőn keresztül használható, a megfelelő jogosultságú felhasználók képesek ülést hirdetni helyszínnel, időponttal, majd erre résztvevőket meghívni, akik erről automatikus értesítést kapnak. Szintén a rendszer feladata a lefolyt ülések jegyzőkönyvei elkészítésének segítése, ezek megőrzése és megfelelő feltételek esetén nyilvánossá tétele.

A rendszer fő hasznélvezője a Pannon Egyetem Hallgatói Önkormányzata lesz, de célom a lehetőség általánosítás, hogy a későbbiekben bárki könnyen és gyorsan az egyedi igényeihez tudja szabni és használatba venni.

\section{Hasonló rendszerek}

Számos jegyzőkönyvvezető rendszer érhető el már magyar nyelven is, de ezek többsége másfélére specializálódott, mint az én rendszerem; például mérési-, verseny vagy vizsgajegyzőkönyv-készítésre.

Három darab ülésjegyzőkönyv-készítő funkcionalitással bíró rendszert találtam, melyek magyar készítésűek:

\begin{itemize}

    \item \emph{AC-TMTR, avagy az Albacomp Testületi Munkát Támogató Rendszer}\\
    Önkormányzatok számára készült, honlapjuk szerint ,,Az AC-TMTR a teljes demokratikus döntéshozatali folyamat támogatására alkalmas az előterjesztések kezelésétől a döntések végrehajtásáig''.\\
    A rendszer tudása messze több, mint jegyzőkönyvkészítés, Lotus Notes alapon működik.
    
    \item \emph{InterMap e-FORTE}\\
    Fő célcsoportja a polgármesteri hivatalok, felső államigazgatás és nagyvállalatok vezetése, ennek tudása is jelentősen meghaladja a jegyzőkönyvkészítést és ülésszervezést.\\
    Webalapú, ASP.NET nyelven készült.
    
    \item \emph{eKÖZIG döntéstámogató rendszer}\\
    Szintén a közigazgatás a fő célcsoport, nagyméretű és bonyolult rendszert eredményezve.\\
    Webalapú, ASP nyelven íródott.
    
\end{itemize}

\section{Rendszerterv}

A rendszer három fő modulra bontható:

\begin{enumerate}
    \item{Felhasználókezelés, authentikáció és authorizáció}
    \item{Ülések hirdetése, kezelése}
    \item{Jegyzőkönyvírás}
\end{enumerate}

\subsection{Felhasználókezelés, authentikáció és authorizáció}

Az alkalmazás zárt, azaz nincs szabad regisztráció, csak az adminisztrátorok által hozható létre új felhasználó, illetve ők tudják a létezőket módosítani. A rendszer a kényelmes, gyors és biztonságos bejelentkezésre a Google OpenID rendszerét használja, így nem kell például jelszavakat biztonságos tárolásáról gondoskodnunk. Minden HÖK tag rendelkezik egy Google Apps fiókkal.

Ennek a modulnak a feladata az OpenID bejelentkezési folyamat megvalósítása, a kapott adatok alapján a felhasználóhoz párosított jogosultsági szint ellenőrzése a rendszer használata során. Szintén ide tartozik a felhasználók adminisztrátorok általi kezelése.

\subsection{Ülések hirdetése, kezelése}

Az ülések rendelkeznek fix számú adattal, mint a rövid leírásuk, időpontjuk és a helyszínük. Megfelelő jogosultsággal rendelkező felhasználók képesek a fenti adatok megadása után ülést hirdetni, meghívottakat megadni. Opcionális adat még az ülés hosszú leírása, itt lehet például megadni a tervezett napirendi pontokat; valamint lehetőség van dokumentumok feltöltésére, ha valami anyagot előzetesen tanulmányozni kell a résztvevőknek.

A rendszer a hirdetett üléseknek létrehoz egy bejegyzést a Google Naptár alkalmazásában, melyre meghívja a résztvevőket, így az bekerül az ő naptárukba is, valamint erről értesítő levelet kapnak.

\subsection{Jegyzőkönyvírás}

A lezajlott ülésekhez készíthető jegyzőkönyv, de természetesen ülés nélkül is lehet jegyzőkönyvet létrehozni (korábbi ülések, vagy ha a szervezés valamely oknál fogva a rendszeren kívül történt).

A jegyzőkönyvek rendelkeznek ugyanazokkal az alapoadatokkal, mint az ülések, valamint tetszőleges számú bejegyzéssel, melyekből három típust különböztetünk meg: napirendi pont, felszólalás valamint szavazás. Mind más adatokkal rendelkezik, a rendszernek kezelnie kell ezt, valamint a bejegyzések egymás közti sorrendjét.

A papír alapú iktatást elősegítendő lehetőség van az elkészült jegyzőkönyvek pdf formátumú exportálására.

Szintén e modul feladata a nyilvános ülések kész jegyzőkönyveinek publikálása, bárki számára elérhetővé tétele.

\end{document}
